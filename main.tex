% !TEX encoding = UTF-8 Unicode
\documentclass[journal]{Imperial_lab_report}


\usepackage{url}
\usepackage{listings}
\usepackage{graphicx}
%\usepackage{subfig}
\usepackage{amsmath}
\hyphenation{}
%\captionsetup{font=small,labelfont={sf,sf}}

\begin{document}

\title{Spectral analysis of light sources using an Michelson interferometer}

\author{Zijian Zhao}%

\markboth{Z. Zhao}%
{Shell \MakeLowercase{\textit{et al.}}:}

\maketitle


\begin{abstract}

We report, 
\end{abstract}


\section{Introduction}
\IEEEPARstart{I}{nterference} 



\section{Theory}
\label{theory}

\begin{figure}[]
\centering
\includegraphics[width=0.4\textwidth]{interference.png}
\caption{Two waves with slightly different phase and wavelength superposing to form the resultant wave at the bottom. Due to the difference in wavelength, beating effect can clearly be observed. The beat period of the combined wave is influenced by the difference in frequency of the composition waves.}
\label{fig:interference}
\end{figure}

\section{Methods}
\label{methods}

\section{Result and analysis}
\label{analysis}
\section{Conclusion}
\label{Conclusion}
Using a Michelson interferometer as a spectrometer, we were able to obtain the interferogram of light sources including the laser, the white/blue LEDs, the tungsten lamp, and the mercury lamp, then obtain its wavelength spectrum via Fourier transform. By analysing the spectral lines and learning their uses in reality we conclude that their functionality is closely related to their frequency components and is essentially determined by their operation principle. For instance, the LEDs are better illuminators than the tungsten lamp due to the fact that unlike tungsten, LEDs has their spectral lines in the visible ranger. Another conclusion is that the mercury lamp is good for decontamination since it emits strong UV radiation at a relatively low temperature. We also observed the Zeeman effect and found that the spectral width increases when the mercury gas discharge lamp is placed in a strong magnetic field. In the end, we proposed some improvements such as better error analysis, and optimised sampling rate. Plus some future research topics related to our experiment such as the stability and repeatability of the interferogram of sources.\\\\


\bibliographystyle{IEEEtran}
\bibliography{Interferometry}


\end{document}
